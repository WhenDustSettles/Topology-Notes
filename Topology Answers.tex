\documentclass{article}
\usepackage[utf8]{inputenc}
\usepackage{amsfonts, amsmath, amssymb}
\usepackage[english]{babel}
% \usepackage{boisik}
\usepackage{amsthm}
\usepackage[margin=0.5in]{geometry}

%\usepackage{tgbonum}
%\usepackage{cmbright}
%\usepackage{textcomp}
\usepackage[object=am]{pgfornament}
\usepackage{tikz-cd}

\theoremstyle{definition}
\newtheorem{definition}{$\boxed{\star}$ Definition}
\newcommand{\tit}[1]{\textit{#1}}
\newtheorem{theorem}{$\boxed{\boxed{\circledast}}$ Theorem}

\newcommand{\nll}[0]{\newline\newline}
\theoremstyle{remark}
\newtheorem*{remark}{Remark}

\theoremstyle{definition}
\newtheorem{corollary}{$ \to $ Corollary}

\theoremstyle{definition}
\newtheorem{proposition}{$\checkmark$ Proposition}

\newenvironment{customproof}[1]{\paragraph{Answer #1:}}{\hfill\ensuremath{\blacksquare}}
\newcommand{\ans}[1]{\textbf{Answer #1}}


\title{Topology \large\\
	 Solution to selected problems }
\author{Animesh Renanse}
\date{\today}
\usepackage{amsthm}

\newcommand{\intrs}{\cap}

\newcommand{\abs}[1]{\left\vert #1\right\vert}
\newcommand{\inv}[1]{#1^{-1}}
\newcommand{\gen}[1]{\left\langle #1\right\rangle}
\newcommand{\order}[1]{\left\vert #1 \right\vert}
\newcommand{\image}[0]{\text{Im }}
\newcommand{\kernel}[0]{\text{Ker }}
\newcommand{\nsg}[0]{\trianglelefteq}
\newcommand{\isomorph}{\cong}
\newcommand{\End}[1]{\text{\textbf{End}}\left(#1\right)}
\newcommand{\Auto}[1]{\text{\textbf{Aut}}\left(#1\right)}
\newcommand{\groupINT}[0]{\mathbb{Z}}

\newcommand{\topo}[1]{\mathcal{#1}}
\newcommand{\subtopo}[2]{\left. #1 \right\vert_{#2}  }
\newcommand{\nbdsys}[2]{\mathfrak{#1}(#2)}
\newcommand{\interior}[1]{#1^\circ}
\newcommand{\boundary}[1]{\partial #1}
\newcommand{\closure}[1]{#1^{\text{cl}}}
\newcommand{\cntnsmap}[2]{\mathscr{C}(#1,#2)}
\newcommand{\path}[0]{\gamma}
\newcommand{\conncomp}[1]{\mathfrak{C}(#1)}
\newcommand{\pathconncomp}[1]{\Pi(#1)}
\newcommand{\union}[0]{\cup}
\newcommand{\bunion}[0]{\bigcup}
\newcommand{\bintrs}[0]{\bigcap}

\renewcommand{\qedsymbol}{\ensuremath{\blacksquare}}


\begin{document}
	
	\maketitle
	\begin{customproof}{2.7.4}
		The question defined any $ O\subseteq M $ as \emph{open} if $ O\in \nbdsys{U}{p} $ for all $ p\in O $. We need to show that this defines a topology $ \topo{M} $ on $ M $. $ \nbdsys{U}{p} $ follows properties 2-5 in Proposition 3.\\
		To check if $ M\in \topo{M} $ : For any $ U \in \nbdsys{U}{p} $ and since $ U\subseteq M $ hence $ M\in \nbdsys{U}{p} $ and hence $ M $ is open.\\
		To check $ \Phi \in \topo{M} $ : For any $ U\in \nbdsys{U}{p} $ then $ \exists \;V\in \nbdsys{U}{p} $ with $ V\subseteq U $. Now since $ U\setminus V \subseteq U \in \nbdsys{U}{p} $. Therefore $ U\setminus V \in \nbdsys{U}{p} $ and since $ V\in \nbdsys{U}{p} $ which implies $ U\setminus V \intrs V = \Phi \in \nbdsys{U}{p} $.\\
		To check if $ O_1,\dots,O_n \in \nbdsys{U}{p}$ then $ O_1\intrs \dots\intrs O_n \in \nbdsys{U}{p} $ : This is just the $ 3^{rd} $ property in Proposition 3.\\
		To check if $ O_1,\dots\in \nbdsys{U}{p}$ then $ O_1 \union \dots \in \nbdsys{U}{p} $ : Denote $ O = \bigcup_{i\in I} O_i $. From the $ 5^{th} $ property, we have for each $ O_i $ a $ V_i \in \nbdsys{U}{p} $ with $ V_i \subseteq O_i $ and $ V_i \in \nbdsys{U}{p} \;\forall\;q\in V_i$. Therefore, $ V  =\bigcup_{i\in I} V_i \subseteq O $. Note that $ V \in \nbdsys{U}{q} \;\forall\;q\in V_i\;\forall\;i\in I$. Since $ p\in V_i $ for any $ i $ therefore $ \bigcup_{i\in I} V_i \in \nbdsys{U}{p} $. Since $ V\subseteq O $, hence $ O = \bigcup_{i\in I} O_i \in \nbdsys{U}{p}  $, proving the first part.\\\\
		Now, to determine the neighborhoods of such a topology. Note the definition of neighborhood : $ U\subseteq M $ is called a neighborhood of $ p\in M $ if $ \exists $ \emph{open} $ O\subseteq M $ such that $ p\in O \subseteq U $. \\
		Consider any $ p\in M $ and an \emph{open} subset containing $ p $, that is $ p\in O \in \nbdsys{U}{p} $. Now since for any $ O \in \nbdsys{U}{p} $, we have $ p\in O $ therefore any \emph{open} $ O\in \nbdsys{U}{p} $ is by definition the neighborhood. Therefore
		\[\nbdsys{U}{p}\subseteq \overline{\nbdsys{U}{p}}\]
		Now consider any $ U \in \overline{\nbdsys{U}{p}} $ which is a neighborhood of $ p $. By definition of neighborhood, $ U $ must contain an \emph{open} subset containing $ p $. Therefore $ \exists \;O^\prime \subseteq U $ such that $ p\in O^\prime $ where $ O^\prime $ is \emph{open}, hence $ O^\prime \in \nbdsys{U}{p} $. Since $ O^\prime \subseteq U $ which implies that $ U\in \nbdsys{U}{p} $ for any $ U\in \overline{\nbdsys{U}{p}} $. Therefore,
		\[\overline{\nbdsys{U}{p}}\subseteq \nbdsys{U}{p}\]
		so that $ \overline{\nbdsys{U}{p}} = \nbdsys{U}{p} $.
	\end{customproof}

\hrulefill

\end{document}

